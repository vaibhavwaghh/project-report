\chapter{Implementation Plan for Next Semester}

\section{Images References} \label{Img}
This is a section with some dummy text and image. This section will be referenced later.

\begin{figure}[h]
    \centering
    \includegraphics{images/SAKEC_logo.png}
    \caption{Image of SAKEC logo}
    \label{fig:sakeclogo}
\end{figure}

\blindtext % remove when adding actual content

The figure \ref{fig:sakeclogo} is the logo of Shah \& Anchor Kutchhi Engineering College

\newpage

\section{Table References}
\begin{table}[h]
    \centering
    \caption{List of Countries, Capital, and Currency} 
    \vspace{12pt}
    \begin{tabular}{|c|c|c|}
        \hline
         Country & Capital & Currency\\
         \hline
         India & New Delhi & Rupee\\
         \hline
         France & Paris & Euro\\
         \hline
         Germany & Berlin & Euro \\
         \hline
    \end{tabular}
   
    \label{tab:country}
\end{table}

Refer \cite{tablegen} for creating tables online in LaTeX.  

\section{Math References} \label{mathrefs}
As mentioned in section \ref{Img}, different elements can be referenced within a document.

\subsection{Powers series} \label{subsection}

\begin{equation} \label{eq:1}
\sum_{i=0}^{\infty} a_i x^i
\end{equation}

Equation \ref{eq:1} is a typical power series.

Refer \cite{Equationgen1} or \cite{Equationgen2} for creating Mathematical Equations online in LaTeX.

\newpage